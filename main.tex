% smaller margins
\usepackage[a4paper, margin=0.3in]{geometry}

% language
\usepackage{babel}
\usepackage{iflang}

\usepackage{multicol} % multiple columns
% column separation
\setlength{\columnsep}{1cm}
% reduce padding around
\setlength{\multicolsep}{1em}

% handle fonts encoding.
% \usepackage[utf8]{inputenc}
% \usepackage[T1]{fontenc}

% do not indent paragraphs
\setlength{\parindent}{0pt}

% include color definitions
\usepackage{xcolor}
\input{colors.tex}

% icons
\usepackage{fontawesome5}
\usepackage{academicons}

% custom section titles
\newcommand\sect[1]{\section*{#1}}
\newcommand\ssect[1]{\subsection*{#1}}

\usepackage{titlesec} % Allows creating custom \section's
\usepackage{tabularx} % Advanced table configurations

\titleformat{\section}{\large\scshape\raggedright}{}{0em}{}[\titlerule] % Section formatting
\titlespacing{\section}{0em}{.8em}{.8em}


\newcommand{\gray}{\color{darkgray!80}} % Custom highlighting for the work experience and education sections
\newcommand{\splitspace}{2cm}

\def\lng{\IfLanguageName{czech}}

\usepackage[pdftitle={Filip Úradník},bookmarks=false,colorlinks=true,allcolors=HueColor]{hyperref}

\babeltags{en=english, cz=czech}

\newcommand{\Praha}{\lng{Praha}{Prague, CZ}}
\newcommand{\Present}{\lng{současnost}{present}}
\newcommand{\CharlesUni}{\lng{Univerzita Karlova}{Charles University}}
\newcommand{\MFF}{\lng{Matematicko-fyzikální fakulta}{Faculty of Mathematics and Physics}}
\newcommand{\point}{\lng{,}{.}}
\newcommand{\rangedash}{\lng{ -- }{--}}

% position, company, date, place
\newenvironment{work}[4] {{\textsc{#1}, \textbf{#2} \hfill \gray\small\itshape #4,\hspace{3pt} #3}\break\small}{}
\newcommand\worksplit{\vspace{.9em}}

\newcommand\skill[2]{ {#1}.\; {\gray\small\itshape #2}}
\newcommand\skillsplit{\vspace{.9em}}
\newcommand\schoolsplit{\vspace{1.5em}}

% school, faculty, place, period
\newenvironment{school}[4]
{{\textbf{#1}\hfill \gray\small\itshape #2,\hspace{3pt} #4\break}\textsc{#3}\hfill\break\small}{}
	% {\tabularx{0.97\linewidth}{>{\raggedleft\scshape}p{\splitspace}X}
	% 	\lng{Škola}{School}       & \textbf{#1} \hfill #3 \\
	% 	\if \relax\detokenize{#2}\relax\else \lng{Fakulta}{Faculty} & #2 \\ \fi
	% 	\lng{Druh studia}{Type of study} & #4          \\
	% 	\lng{Období}{Period}      & #5                 \\
	% }
	% { \endtabularx }

% position, company, date, place
\newenvironment{intro}{\vspace{.3em}\begin{quote}}{\end{quote}\vspace{.3em}}


\begin{document}

\begin{multicols}{2}
	{\fontsize{36}{36}\selectfont\scshape{Filip Úradník}} % Your name at the top

	\columnbreak
	\hfill
	\begin{minipage}{.330\textwidth}
	\small
	\hspace{.4em}\hfill\break
	\texttt{\href{https://matrix.to/\#/@furadnik:matrix.org}{@furadnik:matrix.org}} \\
	\texttt{\href{mailto:filip.uradnik9@gmail.com}{filip.uradnik9@gmail.com}}\; (\lng{osobní email}{personal email}) \\
	\texttt{\href{mailto:uradnik@kam.mff.cuni.cz}{uradnik@kam.mff.cuni.cz}} \; (\lng{školní email}{school email})
	\end{minipage}
\end{multicols}

\vspace{.2em}

\begin{center}
	\small
	\hspace{.5in}
	\faGlobe\ \texttt{\href{https://furadnik.github.io}{furadnik.github.io}}
	\hfill
	\faGithub\ \texttt{\href{https://github.com/furadnik}{furadnik}}                                                                      
	\hfill
	\faLinkedin\ \texttt{\href{https://www.linkedin.com/in/furadnik/}{furadnik}}
	\hfill
	\aiGoogleScholarSquare\ \href{https://scholar.google.com/citations?user=7AvTiqgAAAAJ}{Filip Úradník}
	\hfill
	\aiOrcid\ \texttt{\href{https://orcid.org/0009-0009-3058-6608}{0009-0009-3058-6608}}
	\hspace{.5in}

\end{center}

\begin{intro}
	\lng
	{
		Jsem student magisterského programu Diskrétní modely a algoritmy na Matematicko-fyzikální fakultě Univerzity Karlovy.
		Rád pracuji na projektech na hranici mezi teorií a praxí.
		Ve výzkumu se zajímám o reinforcement learning, algoritmickou teorii her, teorii grafů a kombinatoriku obecně.
		Také se zajímám o free and open source software.
	}{
		I am a master's student of computer science at Charles University in Prague.
		I like to stand at the line between theory and practice.
		My research interests include reinforcement learning, algorithmic game theory, graph theory and combinatorics in general.
		I am also passionate about free and open source software, and all things Neovim.
	}
\end{intro}


\begin{multicols}{2}
\sect{\lng{Pracovní zkušenosti}{Work Experience}}

\begin{work}{REU Student Researcher}{DIMACS}{2024}{NJ, US}
	\lng{Účastnil jsem se programu REU na}
	{I took part in the REU program at} \href{http://dmac.rutgers.edu/}{DIMACS}, \href{https://www.rutgers.edu/}{Rutgers University}.
	\lng{Pod vedením profesorky Jie Gao jsem pracoval na}
	{I worked under the mentorship of Professor Jie Gao on} \emph{\href{https://reu.dimacs.rutgers.edu/~fu37/}{Truth Learning in Social and Adversarial Setting}}.
\end{work}

\worksplit

\begin{work}{Software Developer}{CZ.NIC}{2022\rangedash2024}{\Praha{}}
	\lng{Vývoj CLI pro komunikaci s registrem domén}{I developed a CLI for communication with a domain registry}.\\
	\lng{Dostupné přes systém}{Available on} \href{https://gitlab.nic.cz/fred/eppic}{GitLab}.
\end{work}

\sect{\lng{Dovednosti}{Skills}}

\skill{Linux}{Arch Linux, Networking, CI/CD, Docker}

\skill{Python}{type-hints, dataclasses, functools, itertools, pydantic}

\skill{ML \& Statistics}{numpy, pytorch, keras, scipy, gymnasium}

\skill{C++}{CMake, Catch2, C++17 \lng{a novější}{and newer}}

\LaTeX, Lua, Prolog, Haskell, Rust

\skillsplit

\skill{\lng{Diskrétní matematika}{Discrete Mathematics}}
{\lng{Grafové algoritmy, Regularity lemma, stromová šířka, generující funkce, množinové funkce}
{Graph Algorithms, Regularity Lemma, Treewidth, Generating Functions, Set Functions}}

\skill{\lng{Teoretická informatika}{Theoretical CS}}
{\lng{Parametrizovaná složitost, amortizovaná analýza, složitostní redukce, algoritmická teorie her, kooperativní teorie her}
{Parameterized Complexity, Amortized Analysis, Completeness Reductions, Algorithmic Game Theory, Cooperative Game Theory}}

\skill{\lng{Optimalizace}{Optimization}}
{\lng{Lineární programování, KKT podmínky, semidefinitní programování, (deep) (reinforcement) learning}{Linear Programming, KKT conditions, Semidefinite Programming, (Deep) (Reinforcement) Learning}}

\skillsplit

\skill{\lng{Cizí jazyky}{Languages}}{\lng{}{Czech (native), }\lng{anglický jazyk}{English} (C1), \lng{německý jazyk}{German} (A2)}

\columnbreak
\section{\lng{Vzdělání}{Education}}

\begin{school}{\CharlesUni{}}{\Praha}
{\lng{Magisterské, Diskrétní modely a algoritmy}{Master's degree, Discrete Models and Algorithms}}{2024\rangedash{}\Present{}}
\lng{Průměr známek}{Grade average of} 1\point{}00\lng{}{ (on a \href{https://en.wikipedia.org/wiki/Grading_systems_by_country\#Czech_Republic}{1\rangedash{}4 scale})}.
\end{school}

\schoolsplit

\begin{school}{\CharlesUni{}}{\Praha}
{\lng{Bakalářské, zaměření obecná informatika}{Bachelor's degree, Theoretical Computer Science}}{2021\rangedash{}2024}
\lng{Průměr známek}{Grade average of} 1\point{}03\lng{}{ (on a \href{https://en.wikipedia.org/wiki/Grading_systems_by_country\#Czech_Republic}{1\rangedash{}4 scale})}. \\
\lng{Bakalářská práce na téma}{Thesis on} \href{https://furadnik.github.io/projects/2024_bakalarka.html}{\lng{Balancování prostorové složitosti a nejednoznačnosti superaditivních set funkcí}{Balancing Space Complexity and Ambiguity in Superadditive Set Functions}} \cite{Uradnik2024}, \lng{výsledek: výborně}{grade: excellent}. \\
\lng{Předměty státní zkoušky:}{State exam subjects:} \lng
{Kombinatorika, Diferenciální a integrální počet ve více rozměrech, Pokročilé algoritmy a datové struktury, Pokročilá diskrétní matematika}
{Combinatorics, Multivariable differential \& integral calculus, Advanced algoritms \& data structures, Advanced discrete mathematics}. \lng{Výsledek: výborně}{Grade: excellent}. \\
\lng{Třikrát v řadě získáno stipendium za vynikající studijní výsledky}{Received the scholarship for excellent study results three times in a row}.
\end{school}

\schoolsplit

\begin{school}{Gymnázium Mimoň}{Mimoň, CZ}
	{\lng{Středoškolské s maturitou}{Secondary school with \href{https://en.wikipedia.org/wiki/Matura\#Czech_Republic}{maturita exam}}}{2013\rangedash{}2021}
	Maturita: český jazyk, anglický jazyk, matematika, informatika, matematika rozšiřující.
\end{school}
\end{multicols}

\nocite{*}
\bibliographystyle{unsrt} % We choose the "plain" reference style
\renewcommand{\refname}{\lng{Publikace}{Publications}}
\bibliography{papers_all}

\section{\lng{Další aktivity}{Other Activities}}

\begin{tabularx}{\linewidth}{>{\raggedleft\scshape}p{\splitspace}X}
	2025          & \lng
	{Účastnil jsem se programu Erasmus+ na}
	{I took part in the Erasmus+ program at the} University of Copenhagen. \\
	2024          & \lng
	{Získal jsem \href{https://www.mff.cuni.cz/cs/kam/vyzkum/cena-jirky-matouska}{Cenu Jirky Matouška} za článek  \href{https://www.mff.cuni.cz/cs/kam/vyzkum/cena-jirky-matouska}{\emph{Reducing Optimism Bias in Incomplete Cooperative Games}}~\cite{10.5555/3635637.3663047}.}
	{I received the \href{https://www.mff.cuni.cz/en/kam/research/prize-of-jirka-matousek}{Prize of Jirka Matoušek} for \href{https://www.mff.cuni.cz/cs/kam/vyzkum/cena-jirky-matouska}{\emph{Reducing Optimism Bias in Incomplete Cooperative Games}}~\cite{10.5555/3635637.3663047}.} \\
	2024          & \lng
	{Účastnil jsem se konference \href{https://www.aamas2024-conference.auckland.ac.nz/}{AAMAS 2024} v Aucklandu na Novém Zélandě, kde jsem přednášel o jednom plném článku \cite{10.5555/3635637.3663047} a dvou workshopových článcích.}
	{I attended the \href{https://www.aamas2024-conference.auckland.ac.nz/}{AAMAS 2024 Conference} in Auckland, New Zealand, giving talks on one full paper \cite{10.5555/3635637.3663047}, and two workshop papers.}
	\\
	2024          & \lng
	{Přednášel jsem na \href{https://kam.mff.cuni.cz/~spring/2024/}{Jarní škole kombinatoriky 2024} na téma \emph{učení kooperativních her}.}
	{I gave a talk at the \href{https://kam.mff.cuni.cz/~spring/2024/}{Spring School of Combinatorics 2024} on \emph{Learnability of Cooperative Games}.} \\
	2023          & \lng
	{Přednášel jsem na \href{https://kam.mff.cuni.cz/~spring/2023/}{Jarní škole kombinatoriky 2023} na téma \emph{struktura v kooperativní teorii her}.}
	{I gave a talk at the \href{https://kam.mff.cuni.cz/~spring/2023/}{Spring School of Combinatorics 2023} on \emph{Structure of Cooperative Games}.} \\
	2022\rangedash{}2024     & \lng{Spoluorganizoval jsem programátorskou soutěž \href{https://kasiopea.matfyz.cz}{Kasiopea}.}
	{I co-organized the \href{https://kasiopea.matfyz.cz}{Kasiopea} programming competition.}                                                            \\
	2020\rangedash{}2021 & \lng{Pomáhal jsem vytvořit nové školní stránky \href{https://gymi.cz}{Gymnázia Mimoň}.}
	{I helped to create a new website for \href{https://gymi.cz}{Gymnázium Mimoň}.}                                                                     \\
\end{tabularx}

\end{document}

