% smaller margins
\usepackage[a4paper, margin=0.3in]{geometry}
\pagestyle{empty}

% language
\usepackage{babel}
\usepackage{iflang}

\usepackage{multicol} % multiple columns
% column separation
\setlength{\columnsep}{1cm}
% reduce padding around
\setlength{\multicolsep}{0em}

% handle fonts encoding.
% \usepackage[utf8]{inputenc}
% \usepackage[T1]{fontenc}

% do not indent paragraphs
\setlength{\parindent}{0pt}

% include color definitions
\usepackage{xcolor}
\input{colors.tex}

% icons
\usepackage{fontawesome5}
\usepackage{academicons}
\usepackage[sorting=ydnt,maxbibnames=99,isbn=false]{biblatex}
\addbibresource{papers_all.bib}

\renewcommand*{\mkbibnamefamily}[1]{
\ifitemannotation{highlight}
  {\textbf{#1}}
  {#1}%
\ifitemannotation{first}
  {\textsuperscript{*}}
  {}%
\ifitemannotation{corresponding}
  {$^\dagger$}
  {}%
}%%

% custom section titles
\newcommand\sect[1]{\section*{#1}}
\newcommand\ssect[1]{\subsection*{#1}}

\usepackage{titlesec} % Allows creating custom \section's
\usepackage{tabularx} % Advanced table configurations

\titleformat{\section}{\Large\scshape}{}{0em}{}[\titlerule] % Section formatting
\titlespacing{\section}{0em}{2.1em}{.8em}
\def\np{\textsf{NP}}

% highlight for locations & date ranges
\newcommand{\gray}{\color{darkgray!80}} 

\def\lng{\IfLanguageName{czech}}

\usepackage[pdftitle={Filip Úradník},bookmarks=false,colorlinks=true,allcolors=HueColor]{hyperref}

\babeltags{en=english, cz=czech}

\newcommand{\Praha}{\lng{Praha}{Prague}, \CZ}
\newcommand{\Present}{\lng{současnost}{present}}
\newcommand{\CharlesUni}{\lng{Univerzita Karlova}{Charles University}}
\newcommand{\MFF}{\lng{Matematicko-fyzikální fakulta}{Faculty of Mathematics and Physics}}
\newcommand{\point}{\lng{,}{.}}
\newcommand{\rangedash}{\lng{ -- }{--}}
\newcommand{\CZ}{\lng{ČR}{CZ}}

% position, company, date, place
\newenvironment{work}[4] {{\textbf{#1}, \textsc{#2} \hfill \gray\small\itshape #4,\hspace{3pt} #3}\break\small}{}
\newcommand\worksplit{\vspace{1.5em}}

\newcommand\skill[2]{ \textbf{#1.} {\small\itshape #2}}
\newcommand\skillsplit{\vspace{1.5em}}
\newcommand\schoolsplit{\vspace{1.3em}}

% school, faculty, place, period
\newcommand\schooltitle[4]{{\textbf{#1}\hfill \gray\small\itshape #2,\hspace{3pt} #4\break}\textsc{#3}}
\newenvironment{school}[4]
{\schooltitle{#1}{#2}{#3}{#4}\hfill\break\small}{}

% position, company, date, place
\newenvironment{intro}{\vspace{-.3em}\begin{quote}}{\end{quote}\vspace{.9em}}


\begin{document}

\begin{multicols}{2}
	{\fontsize{36}{36}\selectfont\scshape{Filip Úradník}} % Your name at the top

	\columnbreak
	\hfill
	\begin{minipage}{.330\textwidth}
	\small
	\hspace{.4em}\hfill\break
	\texttt{\href{https://matrix.to/\#/@furadnik:matrix.org}{@furadnik:matrix.org}} \\
	\texttt{\href{mailto:filip.uradnik9@gmail.com}{filip.uradnik9@gmail.com}}\; (\lng{osobní email}{personal email}) \\
	\texttt{\href{mailto:uradnik@kam.mff.cuni.cz}{uradnik@kam.mff.cuni.cz}} \; (\lng{školní email}{school email})
	\end{minipage}
\end{multicols}

\vspace{.5em}

\begin{center}
	\small
	\hspace{.5in}
	\faGlobe\ \texttt{\href{https://furadnik.github.io}{furadnik.github.io}}
	\hfill
	\faGithub\ \texttt{\href{https://github.com/furadnik}{furadnik}}                                                                      
	\hfill
	\faLinkedin\ \texttt{\href{https://www.linkedin.com/in/furadnik/}{furadnik}}
	\hfill
	\aiGoogleScholarSquare\ \href{https://scholar.google.com/citations?user=7AvTiqgAAAAJ}{Filip Úradník}
	\hfill
	\aiOrcid\ \texttt{\href{https://orcid.org/0009-0009-3058-6608}{0009-0009-3058-6608}}
	\hspace{.5in}

\end{center}

\begin{intro}
	\lng
	{
		Jsem student magisterského programu Diskrétní modely a algoritmy na Matematicko-fyzikální fakultě Univerzity Karlovy.
		Rád pracuji na projektech na hranici mezi teorií a praxí.
		Ve výzkumu se zajímám o reinforcement learning, algoritmickou teorii her, teorii grafů a kombinatoriku obecně.
		Také se zajímám o free and open source software.
	}{
		I am a master's student of computer science at Charles University in Prague.
		Though I usually prefer to work on purely theoretical projects, I do also enjoy projects which are part-theory and part-application.
		I am passionate about free and open-source software, and all things Neovim.
	}
\end{intro}


\begin{multicols}{2}
% do not perfectly balance the columns
\raggedcolumns

\sect{\lng{Oblasti výzkumu}{Research Interests}}

Reinforcement learning \& game theory, \\
graph theory \& social networks, \\
combinatorics \& discrete mathematics, \\
cooperative game theory \& set functions.


\sect{\lng{Pracovní zkušenosti}{Work \& Research Experience}}

\begin{work}{Research Intern}{Equilibre}{2025}{\Praha{}}
	\lng{Pomáhal jsem programovat software pro výzkum v oblasti AI.}
	{Helped design \& program software intended for AI research.}
\end{work}

\worksplit

\begin{work}{REU Student Researcher}{DIMACS}{2024}{NJ, US}
	\lng{Účastnil jsem se programu REU na}
	{Participated in REU at} \href{http://dmac.rutgers.edu/}{DIMACS}, \href{https://www.rutgers.edu/}{Rutgers University}.
	\lng{Pracoval jsem pod prof. Jie Gao na}
	{I worked under Prof. Jie Gao on} \emph{\href{https://reu.dimacs.rutgers.edu/~fu37/}{Truth Learning in Social and Adversarial Setting}}.
	\lng{Naše práce vyústila v \href{https://furadnik.github.io/projects/2025_truth_learning}{publikaci} přijatou na konferenci AAMAS 2025}
	{Our work resulted in a \href{https://furadnik.github.io/projects/2025_truth_learning}{publication} accepted to the AAMAS 2025 conference}~\cite{aamas25}.
\end{work}

\worksplit

\begin{work}{Software Developer}{CZ.NIC}{2022\rangedash2024}{\Praha{}}
	\lng{Vývoj CLI v Pythonu pro komunikaci s registrem domén}{Developed a CLI in Python for communication with a domain registry}.
	\lng{Projekt je dostupný přes systém}{The project is available on} \href{https://gitlab.nic.cz/fred/eppic}{GitLab}.
\end{work}

\columnbreak

\sect{\lng{Vzdělání}{Education}}

\schooltitle{University of Copenhagen}{\lng{Kodaň}{Copenhagen}, DK}
{\lng{Erasmus exchange, Computer Science}{Erasmus exchange, Computer Science}}{2025}

\schoolsplit

\begin{school}{\CharlesUni{}}{\Praha}
{\lng{Magisterské, Diskrétní modely a algoritmy}{Master's degree, Discrete Models and Algorithms}}{2024\rangedash{}\Present{}}
\lng{Průměr známek}{Grade average of} 1\point{}00\lng{}{ (on a \href{https://en.wikipedia.org/wiki/Grading_systems_by_country\#Czech_Republic}{1\rangedash{}4 scale})}.
\end{school}

\schoolsplit

\begin{school}{\CharlesUni{}}{\Praha}
{\lng{Bakalářské, zaměření obecná informatika}{Bachelor's degree, Theoretical Computer Science}}{2021\rangedash{}2024}
\lng{Průměr známek}{Grade average of} 1\point{}03\lng{}{ (on a \href{https://en.wikipedia.org/wiki/Grading_systems_by_country\#Czech_Republic}{1\rangedash{}4 scale})}. \\
\lng{Bakalářská práce na téma}{Thesis on} \href{https://furadnik.github.io/projects/2024_bakalarka.html}{\lng{Balancování prostorové složitosti a nejednoznačnosti superaditivních set funkcí}{Balancing Space Complexity and Ambiguity in Superadditive Set Functions}} \cite{Uradnik2024}, \lng{výsledek: výborně}{grade: excellent}. \\
\lng{Předměty státní zkoušky:}{State exam subjects:} \lng
{Kombinatorika, Diferenciální a integrální počet ve více rozměrech, Pokročilé algoritmy a datové struktury, Pokročilá diskrétní matematika}
{Combinatorics, Multi-variable differential \& integral calculus, Advanced algorithms \& data structures, Advanced discrete mathematics}. \lng{Výsledek: výborně}{Grade: excellent}. \\
\lng{Třikrát v řadě získáno stipendium za vynikající studijní výsledky}{Received the scholarship for excellent study results three times in a row}.
\end{school}

\schoolsplit

\schooltitle{Gymnázium Mimoň}{Mimoň, \CZ}
	{\lng{Středoškolské s maturitou}{Secondary school with \href{https://en.wikipedia.org/wiki/Matura\#Czech_Republic}{maturita exam}}}{2013\rangedash{}2021}
\end{multicols}

\nocite{*}
\renewcommand{\refname}{\lng{Publikace}{Publications}}
\defbibnote{mynote}{$^*$ \lng{značí sdílené první autorství}{indicates equal contribution}.}
\printbibliography[prenote=mynote]

\section{\lng{Další aktivity}{Other Activities}}

\begin{tabularx}{\linewidth}{>{\raggedleft\scshape}p{2cm}X}
	2025          & \lng
	{Přednášel jsem na konferenci \href{https://www.mff.cuni.cz/cs/iuuk/akce/konference/stti-2025}{STTI 2025} o \emph{\np-těžkosti problému Truth Learning} \cite{aamas25}.}
	{I gave a talk at the \href{https://www.mff.cuni.cz/cs/iuuk/akce/konference/stti-2025}{STTI 2025} conference on \emph{\np-hardness of Truth Learning} \cite{aamas25}.}
	\\
	2025          & \lng
	{Účastnil jsem se konference \href{https://www.aamas2025.org/}{AAMAS 2025} v Detroitu, USA, kde jsem prezentoval jeden plný článek \cite{aamas25}.}
	{I attended the \href{https://www.aamas2024-conference.auckland.ac.nz/}{AAMAS 2025} conference in Detroit, USA, presenting one full paper \cite{aamas25}.}
	\\
	2024          & \lng
	{Získal jsem \href{https://www.mff.cuni.cz/cs/kam/vyzkum/cena-jirky-matouska}{Cenu Jirky Matouška} za článek  \href{https://furadnik.github.io/projects/2024_reducing}{\emph{Reducing Optimism Bias in Incomplete Cooperative Games}}~\cite{10.5555/3635637.3663047}.}
	{I received the \href{https://www.mff.cuni.cz/en/kam/research/prize-of-jirka-matousek}{Prize of Jirka Matoušek} for \href{https://www.mff.cuni.cz/cs/kam/vyzkum/cena-jirky-matouska}{\emph{Reducing Optimism Bias in Incomplete Cooperative Games}}~\cite{10.5555/3635637.3663047}.} \\
	2024          & \lng
	{Účastnil jsem se konference \href{https://www.aamas2024-conference.auckland.ac.nz/}{AAMAS 2024} v Aucklandu na Novém Zélandu, kde jsem přednášel o jednom plném článku \cite{10.5555/3635637.3663047} a dvou workshopových článcích.}
	{I attended the \href{https://www.aamas2024-conference.auckland.ac.nz/}{AAMAS 2024} conference in Auckland, New Zealand, giving talks on one full paper \cite{10.5555/3635637.3663047}, and two workshop papers.}
	\\
	2024          & \lng
	{Přednášel jsem na \href{https://kam.mff.cuni.cz/~spring/2024/}{Jarní škole kombinatoriky 2024} na téma \emph{učení kooperativních her}.}
	{I gave a talk at the \href{https://kam.mff.cuni.cz/~spring/2024/}{Spring School of Combinatorics 2024} on \emph{Learnability of Cooperative Games}.} \\
	2023          & \lng
	{Přednášel jsem na \href{https://kam.mff.cuni.cz/~spring/2023/}{Jarní škole kombinatoriky 2023} na téma \emph{struktura v kooperativní teorii her}.}
	{I gave a talk at the \href{https://kam.mff.cuni.cz/~spring/2023/}{Spring School of Combinatorics 2023} on \emph{Structure of Cooperative Games}.} \\
	2022\rangedash{}2024     & \lng{Spoluorganizoval jsem programátorskou soutěž \href{https://kasiopea.matfyz.cz}{Kasiopea}.}
	{I co-organized the \href{https://kasiopea.matfyz.cz}{Kasiopea} programming competition.}                                                            \\
	2020\rangedash{}2021 & \lng{Pomáhal jsem vytvořit nové školní stránky \href{https://gymi.cz}{Gymnázia Mimoň}.}
	{I helped to create a new website for \href{https://gymi.cz}{Gymnázium Mimoň}.}                                                                     \\
\end{tabularx}

\vfill
\lng
{Další informace na druhé straně.}
{See the next page for more information.}
% \lng
% {Seznam publikací na další straně.}
% {See the next page for a list of publications.}

\hfill
\href{https://github.com/furadnik/cv/releases/download/latest/uradnik_cv_\lng{cz}{en}.pdf}{
	\color{gray}
	\itshape
	\today
}

\sect{\lng{Dovednosti}{Skills}}

\skill{Python}{
\lng{Pokročilá znalost idiomatického Pythonu včetně funkcionálního stylu (např. \texttt{itertools}, \texttt{functools}) a typových anotací}
{Fluent in idiomatic Python, including type hints and functional style (e.g., \texttt{itertools}, \texttt{functools})}.}

\skill{C++}{\lng{Zkušenost s}{Knowledge of} C++17 \lng{a novějším}{and newer}, unit testing \lng{pomocí}{with} \texttt{Catch2}.}

\textbf{\LaTeX, Lua, Prolog, Haskell, Rust.}

\skillsplit

\skill{Linux}{\lng{Zkušenosti s psaním Dockerfile, CI pipelinami (GitHub Actions, GitLab CI) a základní správou linuxových serverů.}{Experience writing Dockerfiles, CI pipelines (GitHub Actions, GitLab CI), and basic Linux server maintenance.}}

\skill{\lng{ML a statistika}{ML \& Statistics}}{\lng{Znalost knihoven}{Knowledge of} \texttt{numpy}, \texttt{pytorch}, \texttt{gymnasium}, \texttt{scipy} (\lng{užity např. v experimentální části mého výzkumu}{used, e.g., in the experimental parts of my research} \cite{10.5555/3635637.3663047}).}

\skillsplit

\skill{\lng{Diskrétní matematika}{Discrete Mathematics}}
{\lng{Grafové algoritmy, Regularity lemma, stromová šířka, množinové funkce, kooperativní teorie her}
{Graph Algorithms, Set Functions, Cooperative Game Theory, Regularity Lemma, Treewidth}.}

\skill{\lng{Teoretická informatika}{Theoretical CS}}
{\lng{Parametrizovaná složitost, amortizovaná analýza, složitostní redukce, algoritmická teorie her}
{Parameterized Complexity, Amortized Analysis, Completeness Reductions, Algorithmic Game Theory}.}

\skill{\lng{Optimalizace}{Optimization}}
{\lng{Lineární programování, semidefinitní programování, KKT podmínky, (deep) (reinforcement) learning}{Linear Programming, Semidefinite Programming, KKT conditions, (Deep) (Reinforcement) Learning}.}

\skillsplit

\skill{\lng{Cizí jazyky}{Languages}}{\lng{}{Czech (native), }\lng{anglický jazyk}{English} (B2\rangedash{}C1), \lng{německý jazyk}{German} (A2).}

% TODO: put this in a footer
{
	\color{gray}
	\itshape
	\vfill

	Filip Úradník
	\hfill
	2
	\hfill
	\href{https://github.com/furadnik/cv/releases/download/latest/uradnik_cv_\lng{cz}{en}.pdf}{\color{gray}\today}
}

\end{document}

